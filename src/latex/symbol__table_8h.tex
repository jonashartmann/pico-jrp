\hypertarget{symbol__table_8h}{
\section{Referência do Arquivo symbol\_\-table.h}
\label{symbol__table_8h}\index{symbol\_\-table.h@{symbol\_\-table.h}}
}
{\tt \#include $<$stdio.h$>$}\par
\subsection*{Estruturas de Dados}
\begin{CompactItemize}
\item 
struct \hyperlink{structentry__t}{entry\_\-t}
\item 
struct \textbf{\_\-list}
\item 
struct \textbf{\_\-hash}
\end{CompactItemize}
\subsection*{Definições e Macros}
\begin{CompactItemize}
\item 
\hypertarget{symbol__table_8h_032503e76d6f69bc67e99e909c8125da}{
\#define \textbf{TABLE\_\-SIZE}~211}
\label{symbol__table_8h_032503e76d6f69bc67e99e909c8125da}

\item 
\hypertarget{symbol__table_8h_adbbc7b02d94a4c18646813ac8d7dec1}{
\#define \textbf{EOS}~'$\backslash$0'}
\label{symbol__table_8h_adbbc7b02d94a4c18646813ac8d7dec1}

\end{CompactItemize}
\subsection*{Definições de Tipos}
\begin{CompactItemize}
\item 
\hypertarget{symbol__table_8h_7a6c1727d131e8855c087aba7d3ea8a6}{
typedef struct \_\-list \textbf{listNode}}
\label{symbol__table_8h_7a6c1727d131e8855c087aba7d3ea8a6}

\item 
typedef struct \_\-hash \hyperlink{symbol__table_8h_5e6a56afc91001a4b66434b549e3b7bf}{symbol\_\-t}
\begin{CompactList}\small\item\em Encapsulacao de um tipo abstrato que se chamara 'symbol\_\-t'. \item\end{CompactList}\end{CompactItemize}
\subsection*{Funções}
\begin{CompactItemize}
\item 
int \hyperlink{symbol__table_8h_f3a7cbc1fef81d913a02d08e294d0f23}{init\_\-table} (\hyperlink{symbol__table_8h_5e6a56afc91001a4b66434b549e3b7bf}{symbol\_\-t} $\ast$table)
\begin{CompactList}\small\item\em Inicializar a tabela de Hash. \item\end{CompactList}\item 
void \hyperlink{symbol__table_8h_536683ad1a8a2523f70006409a3dfd68}{free\_\-table} (\hyperlink{symbol__table_8h_5e6a56afc91001a4b66434b549e3b7bf}{symbol\_\-t} $\ast$table)
\begin{CompactList}\small\item\em Destruir a tabela de Hash. \item\end{CompactList}\item 
\hyperlink{structentry__t}{entry\_\-t} $\ast$ \hyperlink{symbol__table_8h_a6b03df2f1066890c436b42dd0e53929}{lookup} (\hyperlink{symbol__table_8h_5e6a56afc91001a4b66434b549e3b7bf}{symbol\_\-t} table, char $\ast$name)
\begin{CompactList}\small\item\em Retornar um ponteiro sobre a entrada associada a 'name'. \item\end{CompactList}\item 
int \hyperlink{symbol__table_8h_7394fd3196ac3b53ce94550e4834451d}{insert} (\hyperlink{symbol__table_8h_5e6a56afc91001a4b66434b549e3b7bf}{symbol\_\-t} $\ast$table, \hyperlink{structentry__t}{entry\_\-t} $\ast$entry)
\begin{CompactList}\small\item\em Inserir uma entrada em uma tabela. \item\end{CompactList}\item 
int \hyperlink{symbol__table_8h_4590a83bc564a643c17d45c043547c1a}{print\_\-table} (\hyperlink{symbol__table_8h_5e6a56afc91001a4b66434b549e3b7bf}{symbol\_\-t} table)
\begin{CompactList}\small\item\em Imprimir o conteudo de uma tabela. \item\end{CompactList}\item 
int \hyperlink{symbol__table_8h_c2260bd8c917b424feded79815aa93a4}{print\_\-file\_\-table} (FILE $\ast$out, \hyperlink{symbol__table_8h_5e6a56afc91001a4b66434b549e3b7bf}{symbol\_\-t} table)
\begin{CompactList}\small\item\em Imprimir o conteudo de uma tabela em um arquivo. \item\end{CompactList}\item 
int \hyperlink{symbol__table_8h_d1cd0133d46d90ca1f3681f1cc9466b9}{hashpjw} (char $\ast$s)
\begin{CompactList}\small\item\em Mapear um char$\ast$ para um numero inteiro. \item\end{CompactList}\end{CompactItemize}


\subsection{Descrição Detalhada}
\begin{Desc}
\item[Versão:]1.1 \end{Desc}


\subsection{Definições dos tipos}
\hypertarget{symbol__table_8h_5e6a56afc91001a4b66434b549e3b7bf}{
\index{symbol\_\-table.h@{symbol\_\-table.h}!symbol\_\-t@{symbol\_\-t}}
\index{symbol\_\-t@{symbol\_\-t}!symbol_table.h@{symbol\_\-table.h}}
\subsubsection[{symbol\_\-t}]{\setlength{\rightskip}{0pt plus 5cm}typedef struct \_\-hash {\bf symbol\_\-t}}}
\label{symbol__table_8h_5e6a56afc91001a4b66434b549e3b7bf}


Encapsulacao de um tipo abstrato que se chamara 'symbol\_\-t'. 

Voce deve inserir, entre o 'typedef' e o 'symbol\_\-t', a estrutura de dados abstrata que voce ira implementar. 

\subsection{Funções}
\hypertarget{symbol__table_8h_536683ad1a8a2523f70006409a3dfd68}{
\index{symbol\_\-table.h@{symbol\_\-table.h}!free\_\-table@{free\_\-table}}
\index{free\_\-table@{free\_\-table}!symbol_table.h@{symbol\_\-table.h}}
\subsubsection[{free\_\-table}]{\setlength{\rightskip}{0pt plus 5cm}void free\_\-table ({\bf symbol\_\-t} $\ast$ {\em table})}}
\label{symbol__table_8h_536683ad1a8a2523f70006409a3dfd68}


Destruir a tabela de Hash. 

'free\_\-table' eh o destrutor da estrutura de dados. Deve ser chamado pelo usuario no fim de seu uso de uma tabela de simbolos. \begin{Desc}
\item[Parâmetros:]
\begin{description}
\item[{\em table,uma}]referencia sobre uma tabela de simbolos. \end{description}
\end{Desc}
\hypertarget{symbol__table_8h_d1cd0133d46d90ca1f3681f1cc9466b9}{
\index{symbol\_\-table.h@{symbol\_\-table.h}!hashpjw@{hashpjw}}
\index{hashpjw@{hashpjw}!symbol_table.h@{symbol\_\-table.h}}
\subsubsection[{hashpjw}]{\setlength{\rightskip}{0pt plus 5cm}int hashpjw (char $\ast$ {\em s})}}
\label{symbol__table_8h_d1cd0133d46d90ca1f3681f1cc9466b9}


Mapear um char$\ast$ para um numero inteiro. 

\begin{Desc}
\item[Parâmetros:]
\begin{description}
\item[{\em s,um}]char$\ast$ (string). \end{description}
\end{Desc}
\begin{Desc}
\item[Retorna:]o numero calculado a partir do char$\ast$. \end{Desc}
\hypertarget{symbol__table_8h_f3a7cbc1fef81d913a02d08e294d0f23}{
\index{symbol\_\-table.h@{symbol\_\-table.h}!init\_\-table@{init\_\-table}}
\index{init\_\-table@{init\_\-table}!symbol_table.h@{symbol\_\-table.h}}
\subsubsection[{init\_\-table}]{\setlength{\rightskip}{0pt plus 5cm}int init\_\-table ({\bf symbol\_\-t} $\ast$ {\em table})}}
\label{symbol__table_8h_f3a7cbc1fef81d913a02d08e294d0f23}


Inicializar a tabela de Hash. 

\begin{Desc}
\item[Parâmetros:]
\begin{description}
\item[{\em table,uma}]referencia sobre uma tabela de simbolos. \end{description}
\end{Desc}
\begin{Desc}
\item[Retorna:]o valor 0 se deu certo. \end{Desc}
\hypertarget{symbol__table_8h_7394fd3196ac3b53ce94550e4834451d}{
\index{symbol\_\-table.h@{symbol\_\-table.h}!insert@{insert}}
\index{insert@{insert}!symbol_table.h@{symbol\_\-table.h}}
\subsubsection[{insert}]{\setlength{\rightskip}{0pt plus 5cm}int insert ({\bf symbol\_\-t} $\ast$ {\em table}, \/  {\bf entry\_\-t} $\ast$ {\em entry})}}
\label{symbol__table_8h_7394fd3196ac3b53ce94550e4834451d}


Inserir uma entrada em uma tabela. 

\begin{Desc}
\item[Parâmetros:]
\begin{description}
\item[{\em table,uma}]tabela de simbolos. \item[{\em entry,uma}]entrada. \end{description}
\end{Desc}
\begin{Desc}
\item[Retorna:]um numero negativo se nao se conseguiu efetuar a insercao, zero se deu certo. \end{Desc}
\hypertarget{symbol__table_8h_a6b03df2f1066890c436b42dd0e53929}{
\index{symbol\_\-table.h@{symbol\_\-table.h}!lookup@{lookup}}
\index{lookup@{lookup}!symbol_table.h@{symbol\_\-table.h}}
\subsubsection[{lookup}]{\setlength{\rightskip}{0pt plus 5cm}{\bf entry\_\-t}$\ast$ lookup ({\bf symbol\_\-t} {\em table}, \/  char $\ast$ {\em name})}}
\label{symbol__table_8h_a6b03df2f1066890c436b42dd0e53929}


Retornar um ponteiro sobre a entrada associada a 'name'. 

Essa funcao deve consultar a tabela de simbolos para verificar se se encontra nela uma entrada associada a um char$\ast$ (string) fornecido em entrada. Para a implementacao, sera necessario usar uma funcao que mapeia um char$\ast$ a um numero inteiro. Aconselha-se, por exemplo, consultar o livro do dragao (Aho/Sethi/Ulman), Fig. 7.35 e a funcao HPJW.

\begin{Desc}
\item[Parâmetros:]
\begin{description}
\item[{\em table,uma}]tabela de simbolos. \item[{\em name,um}]char$\ast$ (string). \end{description}
\end{Desc}
\begin{Desc}
\item[Retorna:]um ponteiro sobre a entrada associada a 'name', ou NULL se 'name' nao se encontrou na tabela. \end{Desc}
\hypertarget{symbol__table_8h_c2260bd8c917b424feded79815aa93a4}{
\index{symbol\_\-table.h@{symbol\_\-table.h}!print\_\-file\_\-table@{print\_\-file\_\-table}}
\index{print\_\-file\_\-table@{print\_\-file\_\-table}!symbol_table.h@{symbol\_\-table.h}}
\subsubsection[{print\_\-file\_\-table}]{\setlength{\rightskip}{0pt plus 5cm}int print\_\-file\_\-table (FILE $\ast$ {\em out}, \/  {\bf symbol\_\-t} {\em table})}}
\label{symbol__table_8h_c2260bd8c917b424feded79815aa93a4}


Imprimir o conteudo de uma tabela em um arquivo. 

A formatacao exata e deixada a carga do programador. Deve-se listar todas as entradas contidas na tabela atraves de seu nome (char$\ast$). Deve retornar o numero de entradas na tabela. A saida deve ser dirigida para um arquivo, cujo descritor e passado em parametro.

\begin{Desc}
\item[Parâmetros:]
\begin{description}
\item[{\em out,um}]descrito de arquivo (FILE$\ast$). \item[{\em table,uma}]tabela de simbolos. \end{description}
\end{Desc}
\begin{Desc}
\item[Retorna:]o numero de entradas na tabela. \end{Desc}
\hypertarget{symbol__table_8h_4590a83bc564a643c17d45c043547c1a}{
\index{symbol\_\-table.h@{symbol\_\-table.h}!print\_\-table@{print\_\-table}}
\index{print\_\-table@{print\_\-table}!symbol_table.h@{symbol\_\-table.h}}
\subsubsection[{print\_\-table}]{\setlength{\rightskip}{0pt plus 5cm}int print\_\-table ({\bf symbol\_\-t} {\em table})}}
\label{symbol__table_8h_4590a83bc564a643c17d45c043547c1a}


Imprimir o conteudo de uma tabela. 

A formatacao exata e deixada a carga do programador. Deve-se listar todas as entradas contidas na tabela atraves de seu nome (char$\ast$). Deve retornar o numero de entradas na tabela.

\begin{Desc}
\item[Parâmetros:]
\begin{description}
\item[{\em table,uma}]tabela de simbolos. \end{description}
\end{Desc}
\begin{Desc}
\item[Retorna:]o numero de entradas na tabela. \end{Desc}
